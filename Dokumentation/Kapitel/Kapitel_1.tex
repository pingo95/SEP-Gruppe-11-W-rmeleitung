\chapter{Vorwort}

Dieses Dokument erfasst den Verlauf des \emph{Simulationssoftwareentwicklungspraktikums} des Studiengangs Computational Engineering Science an der RWTH Aachen im Jahr 2015. \\

\noindent
Gegenüber der Präsentation wurden noch folgende Dinge ergänzt:\\
Eine modifizierte Version des Use-Case-Diagramms wurde im Abschnitt \ref{Anwendungsfalldiagramm} \emph{Anwendungsfalldiagramm} auf Seite \pageref{Use Case Diagramm2} ergänzt. Desweiteren wurden zusätzlich eine Visualisierung der Optimierungsergebnisse hinsichtlich der örtlichen Verteilung der Temperaturleitkoeffizienten sowie die Möglichkeit zur Übernahme der Ergebnisse in das aktuelle Simulationsexperiment implementiert.


\section{Aufgabenstellung und Struktur des Dokument}

Aufgabe ist es eine Software zur Simulation eines 2D-Wärmeleitungsproblems zu erstellen. Die Struktur des Dokumentes orientiert sich am Entwicklungsprozess der Software, beginnt mit der Analyse des Problems, fährt fort mit dem Entwurf der Software, woran sich die Benutzer- und zuletzt die Entwicklerdokumentation anschließt.

\section{Projektmanagement}

Die Problemanalyse sowie der Entwurf der Software erfolgten in Gruppenarbeit. Während der Implementierungsphase wurden die Aufgaben, grob an der Paketstruktur orientiert, an die einzelnen Mitglieder verteilt. Alle sich anschließenden Schritte wurden gemeinsam bearbeitet.

\section{Lob und Kritik}

Ein ausdrückliches Dankeschön möchten wir Markus Towara aussprechen, der uns stets umgehend bei Fragen und Problemen, ob persönlich oder per E-Mail, geholfen hat.
Als Verbesserungsvorschlag: Ein beispielhafter Datensatz experimenteller Daten könnte gestellt werden, um das Parameter-Fitting besser nachvollziehen/verifizieren zu können.